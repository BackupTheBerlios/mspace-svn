\documentclass{article}
\usepackage[utf8]{inputenc}
\usepackage[activeacute,spanish]{babel}
\author{Néstor Salceda}
\title{MSpace ComponentModel 0.1}
\date{\today}
\begin{document}
\maketitle
\section{Propiedades y requisitos que cada componente software debe cumplir:}
\begin{itemize}
\item Cumplimiento del paradigma de la programación orientada a componentes
software.
\item Separación clara entre modelo, vista y controlador.
\item Independencia entre componentes.
\item ¿Variación del comportamiento de un componente según una definición
exterior, sin modificar el código fuente?
\item Estandarización de la nomenclatura y de la jerarquia de clases interna de
un componente.
\end{itemize}
\section{Propiedades de la framework para la versión 0.1:}
\begin{itemize}
\item Localización de los componentes por un nombre e identificador único.
\item Ejecución de los casos de uso de un componente dinámicamente.
\item Posiblidad de redirección a la vista:
\begin{itemize}
\item Indicar manejo de la vista.
\item Indicar que método rebe ser ejecutado en la vista como respuesta a un caso
de uso.
\item Indicar qué instancia de la vista se debe utilizar.
\end{itemize}
\item Posibilidad de llamar a métodos no bloqueantes.
\item Concentración en el manejo de excepciones.
\end{itemize}
\section{Explicación de cada una de las propiedades:}
\subsection{Localización de componentes por nombre:}
Para tener acceso a los servicios que nos brinda un componente se debe conseguir
una instancia de dicho componente, para conseguir esta instancia se le pasará el
nombre (identificador) de un componente.
Una vez conseguida la instancia se provee una interface común para el acceso a
los servicios que ofrece el componente.
\subsection{Ejecución de los métodos dinámicamente:}
Resolución de los métodos a ejecutar en tiempo de ejecución, pudiendo variar el
comportamiento de la ejecución de los casos de uso; según los parámetros
pasados.
\subsection{Redirección a la vista:}
Cuando ejecutamos un caso de uso del componente, se puede optar por redirigir a
la vista la salida de ejecución de dicho caso de uso.  Al igual que a veces no
tiene porqué ser necesario el redirigirlo a una vista.  Y también se debe poder
especificar a qué vista queremos que se le redirija.
\subsection{Métodos no bloqueantes:}
Se trata de no esperar a que un método finalice su ejecución para poder
proseguir con la siguiente ejecución.  Un ejemplo claro es la ejecución de un
método en segundo plano.
\subsection{Manejo de excepciones:}
A través de un controlador de excepciones, todas las excepciones producidas de
la ejecución de un método serán redirigidas a un controlador de excepciones y
desde ahí exisitirá un método común de procesado de las excepciones.
\section{Posibles propiedades de la framework en versiones futuras:}
\begin{itemize}
\item Acceso a los casos de uso mediante roles.
\item Ejecución de los métodos y las respuestas de manera transaccional.
\item Métodos virtuales.
\item Preejecución, postejecución e intercepción de métodos.
\end{itemize}
\end{document}
