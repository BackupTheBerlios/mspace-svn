\documentclass[a4paper]{article}
\usepackage[utf8]{inputenc}
\usepackage[activeacute,spanish]{babel}
\author{Néstor Salceda}
\title{Jerarquía del directorio src.}
\begin{document}
\section{Introducción:}
Cuando comenzamos a escribir algo de código en alguno de nuestros proyectos
anteriores, encontramos algunos problemas a la hora de ponernos de acuerdo para
nombrar nuestros ficheros y crear la jerarquía de directorios.  Esto no
representa un problema si se trata de un proyecto pequeño; pero si se trata de
un proyecto con varios componentes, sería recomendable saber qué es lo que hace
cada componente y saber que partes va a tener sin tener que perder ese tiempo en
mirar el código fuente, utilizando nombres descriptivos.

Ahora, tenemos entre manos quizás el primer proyecto grande que sabemos que
puede ser reutilizado en algún otro proyecto (common vfs).  Y es bastante
importante que tengamos estas cosas bien claras.

Como herramientas, en el subversion de mspace tenemos ya creado un build para
NAnt, en scripts/nant.xml.  Ese build es una implementación de referencia para
la estructura que voy a detallar ahora.

\section{Descripción:}
Ahora que ya conocemos el problema  que hemos tenido; mi propuesta es la
siguiente.  Es una propuesta que ya he utilizado alguna vez con resultados
bastante buenos, a mí personalmente me gusta.  También expondré las
modificaciones que serían factibles de realizar.
\subsection{Proyecto:}
Utilizaremos proyecto como el programa que vamos a realizar, ese proyecto puede
tener 1 o más componentes.  Y cada proyecto debe tener un nombre lo
suficientemente descriptivo de lo que será la aplicación.  Sin tener miedo a
poner nombres largos.
\subsection{Comunes a proyecto:}
Así pues, tendremos el directorio raiz, que será el user.dir.  Ese directorio
contendrá otros directorios, concretamente: lib, src, resources
y luego además existirá un bin y un doc que se irán creando conforme utilizemos
nant para compilar nuestro proyecto completo.
\subsubsection{lib}
Lib contendrá otras librerías externas de las que dependa nuestro proyecto, si
por ejemplo vamos a utilizar la libreria log4net, deberemos copiar el ensamblado
a lib.
Otra opción que se barajará es si esa librería está instalada en el gac, como un
strong assembly, pero eso ya es otro tema.  En principio en esta IR no se da
soporte de integración con el gac.
Si ademas en lib vamos a utilizar otros mspace components, deberán ser copiados
a lib/components.  Con su ensamblado y además otro fichero que describa el
interface del component (falta mucho por discutir sobre como sacar esos
interfaces).
\subsubsection{resources}
Aquí vendrán todo aquello que sean recursos para la aplicación, todavía el build
de nant no soporta la inclusión de resources en el ensamblado, pero en la
próxima versión si que lo hará.
Así pues podemos tomar como recursos, todo aquello que sean iconos, mensajes de
i8n, scripts para ejecución de programas externos, logs, ficheros de
configuración ...
\subsubsection{src}
El corazón del proyecto, aqui se guardará el código fuente de los componentes de
la aplicación.  Más tarde pasaré a hablar sobre el esbozo de modelo de
componentes que he pensado y propongo utilizar en los componentes mspace, para
alguien que conozca la framework guasaj, le resultará bastante similar.
\subsubsection{bin}
Aqui se guardaran los binarios de la aplicación, así como los pares de claves
para firmar un ensamblado.
\subsubsection{doc}
Parece sencillo pensar que es lo que habrá aqui.
\subsection{src}
De nuevo volvemos a src, aqui se explicará como hay que organizar nuestro código
fuente.
Para no utilizar un sistema como el que utiliza java, es decir, que para cada
paquete lleva implicito un directorio, dado que en .NET eso no es una
restricción y además podemos hacer más accesible nuestro código fuente.
\subsubsection{appname}
Lo primero que encontraremos será un directorio con el nombre de la aplicación.
Ese directorio contendrá \emph{solamente} el lanzador de la aplicación, que
sería conveniente llamarlo Launcher.cs y la clase debería ser Launcher. Así a
secas, y en su main debería hacer una llamada al InitApp del maincomponent.
\subsubsection{appname.Components}
Este directorio contendrá el código fuente de los componentes que hayamos
escrito a la hora de escribir la aplicación.  Solo contendrá el nombre del
componente que está escrito.  Si por ejemplo quiero un componente que sea un
conector TCPIP, y lo escribo ahi, deberia llamarse tcpipconnector (o algo así
similar).
\subsection{Jerarquia en un componente:}
Dentro de cada componente y adaptandonos ya un poco al modelo de componentes,
tenemos los siguiente subdirectorios: Bo, Dao, Exception, Factory, Form,
Interfaces, Resources, Vo
\subsubsection{Bo}
Aquí se implementará toda la lógica de negocio de nuestro componente, no tiene
porque contener sólo una sola clase. Pero deberá contener un punto de entrada al
modelo que contendrá los casos de uso de nuestra aplicación, y esos casos de uso
serán los que deben ser implementados de nuestro interface.  Como se implementen
esos casos de uso, es tarea del programador del componente.  Puede utilizar más
clases que estén dentro de Bo (la idea es tener todo el business object (Bo)) en
este namespace.
\subsubsection{Dao}
Aquí estará todo aquello que sea un Data Access Object, es decir, objetos que
sean para acceder a datos; esto es conexión con una base de datos, clases para
leer xml.  \emph{No se debe acceder a Dao desde el Form!!!}
\subsubsection{Exception}
Aquí se meteran las excepciones que definamos nosotros para nuestro Bo y guasaj
implementa también un gestor de esas excepciones con el método processException
(Exception exception).  El gestor de excepciones es útil para tener centralizado
todo el código de nuestras excepciones y saber dónde se gestionaran los errores.
\subsubsection{Factory}
Aqui estarán las factorias que utilizaremos para crear los objetos, existen 2
filosofias que seguir acá (las que yo propongo).  De entrada, como a mi me han
dicho y explicado:  
\begin{quote}
Todo es un singleton hasta que se demuestre lo contrario.
\end{quote}
Es preferible hacer todo un singleton y si necesitamos varias instancias ya no
deberá ser un singleton (evidentemente). 
Y la otra es que cuanto menos llamemos a los constructores dentro del BO mejor,
que mejor.  Es preferible utilizar factorías para construir los objetos, dado
que estos pueden ser extremadamente complejos y que a veces podemos necesitar
cambiar un objeto en tiempo de ejecución. (Esto se aplica bastante bien a la
hora de crear y utilizar un Strategy).
\subsubsection{Form}
Aqui se ubicará todo aquello que sea parte de la vista.  No importa la framework
que vayamos a utilizar, nos dará lo mismo utilizar GTK o Windows.Forms o
wxWidgets (no han sido consideradas las WebApps).  Tiene que ser 
transparente y el modelo de componentes para ello
obliga a implementar un ViewDispatcher para hacer esto de forma transparente.
\subsubsection{Interfaces}
Aqui se ubicaran los interfaces que se deban implementar en las clases del Bo.
Ya son de sobras conocidas las ventajas que aportan los interfaces a nuestras
jerarquías de clases.
\subsubsection{Resources}
Los recursos \emph{propios} de cada componente, es como el resources de más
arriba, pero con un ámbito más reducido a cada componente.  Probablemente
también se empaqueten en otro ensamblado los resources.
\subsubsection{VO}
El value object es uno de los patrones más sencillos y poderosos que he
conocido.  Para los que hayan programado en Java a este nivel, les resultará muy
similar a un Bean, pero bastante más sencillo.
Simplemente estas clases son valores, es decir, estas clases no contienen
acciones ni método, sólo propiedades.
Un ejemplo claro puede ser la necesidad de crear una configuración y llamar a
los métodos del Bo con una configuración especifica según como haya sido
configurado.  Asi pues utilizamos una clase ConfigurationVO, que contendrá por
ejemplo un valor string Nombre getter; setter; y según sea ese valor se comportará
de uno u otro modo la aplicación.
Para rellenar los valores de aqui, es útil crearse un Dao que especificamente
rellene el Vo.  Por ejemplo si tenemos que utilizar una serialización XML de
esos valores, es fácil hacer que se serialice o que acceda a los campos del xml
el dao y no el Vo.
\appendix
\section{Sobre el modelo de componentes:}
Aqui voy a dar unas pequeñas pinceladas sobre el modelo de componentes que vamos
a utilizar para evitar el desacople.  Está muy basado en Guasaj framework para
Java, realmente yo cuando programo con Java ya sólo lo hago con Guasaj; por las
ventajas que tiene, pero por supuesto también tiene carencias que he visto y que
me gustaría solucionar.

La primera pregunta es: .NET ya tiene su propio modelo de componentes, ¿porqué
no lo utilizas?  El modelo de componentes de .NET es un buen modelo, pero quizás
sea demasiado complejo para poder llevar a cabo estas aplicaciones, no sólo es
un modelo de componentes, sino que también es un contenedor y el tema de los
contextos no creo que sea necesario para realizar estas aplicaciones. También
decir, que exactamente lo que se va a utilizar no es un modelo de componentes
completo, sería algo así mas parecido a una framework para el modelado de
componentes y la organizacion del código, ayudando al desacoplo entre
componentes y manteniendo las demás características de una programación
orientada a componentes.  Compatibilidad binaria, gozando de independencia de
lenguajes, un mspace component debería ser accesible desde cualquier lenguaje
.NET, y por supuesto cada instancia del componente es única, cada componente
debe ser un singleton (para variar).

No soy un experto en guasaj, pero he podido ver un poquito las ideas que esa
framework conlleva, cada componente llevará unos casos de uso, y la clase que
implementa esos casos de uso debe heredar de DefaultComponentHandler (en
guasaj).  Un componente se llama desde un configuration con el método
getComponentByName (string cmpName); una vez conseguido el
DefaultComponentHanlder del componente, utiliza el patrón executor para ejecutar
los casos de uso, que por reflexión accederá a los interfaces y lo ejecutará,
además de ejecutar una response a la vista donde haya sido definida en el
interface.  Es decir, el flujo de ejecución es el siguiente:  Conseguimos el
componente, y hacemos un .execute del método con los parametros que sean
necesarios ese caso de uso, en su interface xml lleva un response, que será
ejecutado a continuación del caso de uso y ese response debe ser implementado en
la clase que le digamos al interface (si no Exception)  además cada componente
lleva asociado un ExceptionManager que se encargará de localizar las
excepciones y de tratarlas.  Así pues conseguimos un desacople casi perfecto 
(por no decir perfecto) entre componentes y por supuesto un caso de uso puede
responder a una vista, o a otra; incluso crear nuevas instancias de la vista con
el ViewDispatcher y además asociado a ello control de concurrencia sobre los
componentes.

Así pues, tendremos también un config indicando los componentes que existen y la
localización de sus interfaces, logrando así tener nuestro ensamblado del
componente por un lado, y por el otro la configuración. 

Esto puede tener también un problema, que es que un mal uso del fichero xml
puede llevar a errores, por este motivo el xml debe ser bastante sencillo para
localizar los errores que nos pueda dar.

También tiene una restricción fuerte, pero espero que sea solucionada pronto con
el vfs; es la posibilidad de invocar componentes que estén en otra máquina; pero
esto es bastante más complejo y si queremos hacer esto no deberíamos utilizar
esta framework, sino un modelo de componentes completo.  Dado que nosotros no
creo que vayamos a utilizar esto dado que las aplicaciones que vamos a realizar
tienen un ámbito de aplicaciones de escritorio.

Otro problema que he visto es la notificacion de eventos, pero creo que esto es
más por carencias del propio lenguaje Java frente a .NET y tener event como un
tipo propio del lenguaje.  Con este event podemos hacer estas notificaciones de
una buena manera, pero \emph{no se deben definir eventos en el interface!!} debe
ser algo más interno al componente.

Finalmente el problema puede ser la redireccion a varias vistas lo cual el xml
debe proveer un medio para decir a que vistas debe ser dirigida la response de
un caso de uso.  Lo cual, ampliando la definición del xml debe ser una tarea
sencilla.

\section{Requerimientos}
La técnica que se va a utilizar conlleva un uso bastante grande de reflexión
esto no es ningún problema.

El xml debe ser tratado de un modo rápido, pero dado que se debe permitir que
los componentes no sean grandes, sino que sean algo parecido a piezas de un
puzzle en nuestra aplicación, el poder utilizar DOM para parsear el xml no es
ningún problema.

Además es preferible en todas las clases que estén fuera del ComponentHandler,
hacer un get del componente para realizar las acciones; es decir, si en el form
tengo un boton que lo clicko y debe realizar un caso de uso del componente, no
llamar al Bo; sino que obtener el componente y utilizar el executor que lleva.
\end{document}
